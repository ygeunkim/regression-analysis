\documentclass[]{book}
\usepackage{lmodern}
\usepackage{amssymb,amsmath}
\usepackage{ifxetex,ifluatex}
\usepackage{fixltx2e} % provides \textsubscript
\ifnum 0\ifxetex 1\fi\ifluatex 1\fi=0 % if pdftex
  \usepackage[T1]{fontenc}
  \usepackage[utf8]{inputenc}
\else % if luatex or xelatex
  \ifxetex
    \usepackage{mathspec}
  \else
    \usepackage{fontspec}
  \fi
  \defaultfontfeatures{Ligatures=TeX,Scale=MatchLowercase}
\fi
% use upquote if available, for straight quotes in verbatim environments
\IfFileExists{upquote.sty}{\usepackage{upquote}}{}
% use microtype if available
\IfFileExists{microtype.sty}{%
\usepackage{microtype}
\UseMicrotypeSet[protrusion]{basicmath} % disable protrusion for tt fonts
}{}
\usepackage[margin=1in]{geometry}
\usepackage{hyperref}
\hypersetup{unicode=true,
            pdftitle={R Lab for Regression Analysis},
            pdfborder={0 0 0},
            breaklinks=true}
\urlstyle{same}  % don't use monospace font for urls
\usepackage{natbib}
\bibliographystyle{apalike}
\usepackage{color}
\usepackage{fancyvrb}
\newcommand{\VerbBar}{|}
\newcommand{\VERB}{\Verb[commandchars=\\\{\}]}
\DefineVerbatimEnvironment{Highlighting}{Verbatim}{commandchars=\\\{\}}
% Add ',fontsize=\small' for more characters per line
\usepackage{framed}
\definecolor{shadecolor}{RGB}{248,248,248}
\newenvironment{Shaded}{\begin{snugshade}}{\end{snugshade}}
\newcommand{\KeywordTok}[1]{\textcolor[rgb]{0.13,0.29,0.53}{\textbf{#1}}}
\newcommand{\DataTypeTok}[1]{\textcolor[rgb]{0.13,0.29,0.53}{#1}}
\newcommand{\DecValTok}[1]{\textcolor[rgb]{0.00,0.00,0.81}{#1}}
\newcommand{\BaseNTok}[1]{\textcolor[rgb]{0.00,0.00,0.81}{#1}}
\newcommand{\FloatTok}[1]{\textcolor[rgb]{0.00,0.00,0.81}{#1}}
\newcommand{\ConstantTok}[1]{\textcolor[rgb]{0.00,0.00,0.00}{#1}}
\newcommand{\CharTok}[1]{\textcolor[rgb]{0.31,0.60,0.02}{#1}}
\newcommand{\SpecialCharTok}[1]{\textcolor[rgb]{0.00,0.00,0.00}{#1}}
\newcommand{\StringTok}[1]{\textcolor[rgb]{0.31,0.60,0.02}{#1}}
\newcommand{\VerbatimStringTok}[1]{\textcolor[rgb]{0.31,0.60,0.02}{#1}}
\newcommand{\SpecialStringTok}[1]{\textcolor[rgb]{0.31,0.60,0.02}{#1}}
\newcommand{\ImportTok}[1]{#1}
\newcommand{\CommentTok}[1]{\textcolor[rgb]{0.56,0.35,0.01}{\textit{#1}}}
\newcommand{\DocumentationTok}[1]{\textcolor[rgb]{0.56,0.35,0.01}{\textbf{\textit{#1}}}}
\newcommand{\AnnotationTok}[1]{\textcolor[rgb]{0.56,0.35,0.01}{\textbf{\textit{#1}}}}
\newcommand{\CommentVarTok}[1]{\textcolor[rgb]{0.56,0.35,0.01}{\textbf{\textit{#1}}}}
\newcommand{\OtherTok}[1]{\textcolor[rgb]{0.56,0.35,0.01}{#1}}
\newcommand{\FunctionTok}[1]{\textcolor[rgb]{0.00,0.00,0.00}{#1}}
\newcommand{\VariableTok}[1]{\textcolor[rgb]{0.00,0.00,0.00}{#1}}
\newcommand{\ControlFlowTok}[1]{\textcolor[rgb]{0.13,0.29,0.53}{\textbf{#1}}}
\newcommand{\OperatorTok}[1]{\textcolor[rgb]{0.81,0.36,0.00}{\textbf{#1}}}
\newcommand{\BuiltInTok}[1]{#1}
\newcommand{\ExtensionTok}[1]{#1}
\newcommand{\PreprocessorTok}[1]{\textcolor[rgb]{0.56,0.35,0.01}{\textit{#1}}}
\newcommand{\AttributeTok}[1]{\textcolor[rgb]{0.77,0.63,0.00}{#1}}
\newcommand{\RegionMarkerTok}[1]{#1}
\newcommand{\InformationTok}[1]{\textcolor[rgb]{0.56,0.35,0.01}{\textbf{\textit{#1}}}}
\newcommand{\WarningTok}[1]{\textcolor[rgb]{0.56,0.35,0.01}{\textbf{\textit{#1}}}}
\newcommand{\AlertTok}[1]{\textcolor[rgb]{0.94,0.16,0.16}{#1}}
\newcommand{\ErrorTok}[1]{\textcolor[rgb]{0.64,0.00,0.00}{\textbf{#1}}}
\newcommand{\NormalTok}[1]{#1}
\usepackage{longtable,booktabs}
\usepackage{graphicx,grffile}
\makeatletter
\def\maxwidth{\ifdim\Gin@nat@width>\linewidth\linewidth\else\Gin@nat@width\fi}
\def\maxheight{\ifdim\Gin@nat@height>\textheight\textheight\else\Gin@nat@height\fi}
\makeatother
% Scale images if necessary, so that they will not overflow the page
% margins by default, and it is still possible to overwrite the defaults
% using explicit options in \includegraphics[width, height, ...]{}
\setkeys{Gin}{width=\maxwidth,height=\maxheight,keepaspectratio}
\IfFileExists{parskip.sty}{%
\usepackage{parskip}
}{% else
\setlength{\parindent}{0pt}
\setlength{\parskip}{6pt plus 2pt minus 1pt}
}
\setlength{\emergencystretch}{3em}  % prevent overfull lines
\providecommand{\tightlist}{%
  \setlength{\itemsep}{0pt}\setlength{\parskip}{0pt}}
\setcounter{secnumdepth}{5}
% Redefines (sub)paragraphs to behave more like sections
\ifx\paragraph\undefined\else
\let\oldparagraph\paragraph
\renewcommand{\paragraph}[1]{\oldparagraph{#1}\mbox{}}
\fi
\ifx\subparagraph\undefined\else
\let\oldsubparagraph\subparagraph
\renewcommand{\subparagraph}[1]{\oldsubparagraph{#1}\mbox{}}
\fi

%%% Use protect on footnotes to avoid problems with footnotes in titles
\let\rmarkdownfootnote\footnote%
\def\footnote{\protect\rmarkdownfootnote}

%%% Change title format to be more compact
\usepackage{titling}

% Create subtitle command for use in maketitle
\providecommand{\subtitle}[1]{
  \posttitle{
    \begin{center}\large#1\end{center}
    }
}

\setlength{\droptitle}{-2em}

  \title{R Lab for Regression Analysis}
    \pretitle{\vspace{\droptitle}\centering\huge}
  \posttitle{\par}
    \author{Young-geun Kim\\
Department of Statistics, SKKU\\
\href{mailto: dudrms33@g.skku.edu}{dudrms33@g.skku.edu}}
    \preauthor{\centering\large\emph}
  \postauthor{\par}
      \predate{\centering\large\emph}
  \postdate{\par}
    \date{04 Apr, 2019}

\usepackage{booktabs}
\usepackage{float}
\usepackage{pdfpages}

\let\oldmaketitle\maketitle
\AtBeginDocument{\let\maketitle\relax}

\usepackage{amsthm}
\newtheorem{theorem}{Theorem}[chapter]
\newtheorem{lemma}{Lemma}[chapter]
\newtheorem{corollary}{Corollary}[chapter]
\newtheorem{proposition}{Proposition}[chapter]
\newtheorem{conjecture}{Conjecture}[chapter]
\theoremstyle{definition}
\newtheorem{definition}{Definition}[chapter]
\theoremstyle{definition}
\newtheorem{example}{Example}[chapter]
\theoremstyle{definition}
\newtheorem{exercise}{Exercise}[chapter]
\theoremstyle{remark}
\newtheorem*{remark}{Remark}
\newtheorem*{solution}{Solution}
\let\BeginKnitrBlock\begin \let\EndKnitrBlock\end
\begin{document}
\maketitle

\begin{titlepage}
  \includepdf{cover.pdf}
\end{titlepage}

\let\maketitle\oldmaketitle
\maketitle

{
\setcounter{tocdepth}{1}
\tableofcontents
}
\chapter*{Welcome}\label{welcome}
\addcontentsline{toc}{chapter}{Welcome}

 This book aims at covering materials of regression analysis. Also,
there will be R programming for regression.

\chapter*{Linear Regression Analysis}\label{linear-regression-analysis}
\addcontentsline{toc}{chapter}{Linear Regression Analysis}

\begin{Shaded}
\begin{Highlighting}[]
\KeywordTok{data}\NormalTok{(BioOxyDemand, }\DataTypeTok{package =} \StringTok{"MPV"}\NormalTok{)}
\NormalTok{(BioOxyDemand <-}
\StringTok{  }\NormalTok{BioOxyDemand }\OperatorTok\StringTok{ }
\StringTok{  }\KeywordTok{tbl_df}\NormalTok{())}
\end{Highlighting}
\end{Shaded}

\begin{verbatim}
# A tibble: 14 x 2
       x     y
   <int> <int>
 1     3     4
 2     8     7
 3    10     8
 4    11     8
 5    13    10
 6    16    11
 7    27    16
 8    30    26
 9    35    21
10    37     9
11    38    31
12    44    30
13   103    75
14   142    90
\end{verbatim}

\section*{Relation}\label{relation}
\addcontentsline{toc}{section}{Relation}

We wonder how \texttt{x} affects \texttt{y}, especially linearly.

\begin{itemize}
\tightlist
\item
  Functional relation: mathematical equation,
  \[y = \beta_0 + \beta_1 x\]
\item
  Statistical relation: embeded with noise
\end{itemize}

So we try to estimate

\[y = \beta_0 + \beta_1 x + \epsilon\]

\begin{Shaded}
\begin{Highlighting}[]
\NormalTok{BioOxyDemand }\OperatorTok\StringTok{ }
\StringTok{  }\KeywordTok{ggplot}\NormalTok{(}\KeywordTok{aes}\NormalTok{(x, y)) }\OperatorTok{+}
\StringTok{  }\KeywordTok{geom_point}\NormalTok{()}
\end{Highlighting}
\end{Shaded}

\begin{center}\includegraphics[width=0.7\linewidth]{regression-analysis_files/figure-latex/unnamed-chunk-3-1} \end{center}

Looking just with the eyes, we can see the linear relationship.
Regression analysis estimates the relationship statistically.

\begin{Shaded}
\begin{Highlighting}[]
\NormalTok{BioOxyDemand }\OperatorTok\StringTok{ }
\StringTok{  }\KeywordTok{ggplot}\NormalTok{(}\KeywordTok{aes}\NormalTok{(x, y)) }\OperatorTok{+}
\StringTok{  }\KeywordTok{geom_smooth}\NormalTok{(}\DataTypeTok{method =} \StringTok{"lm"}\NormalTok{) }\OperatorTok{+}
\StringTok{  }\KeywordTok{geom_point}\NormalTok{()}
\end{Highlighting}
\end{Shaded}

\begin{center}\includegraphics[width=0.7\linewidth]{regression-analysis_files/figure-latex/unnamed-chunk-4-1} \end{center}

\chapter{Simple Linear Regression}\label{simple}

\section{Model}\label{model}

\begin{Shaded}
\begin{Highlighting}[]
\NormalTok{delv <-}\StringTok{ }\NormalTok{MPV}\OperatorTok{::}\NormalTok{p2.}\DecValTok{9} \OperatorTok\StringTok{ }\KeywordTok{tbl_df}\NormalTok{()}
\end{Highlighting}
\end{Shaded}

\begin{Shaded}
\begin{Highlighting}[]
\NormalTok{delv }\OperatorTok\StringTok{ }
\StringTok{  }\KeywordTok{ggplot}\NormalTok{(}\KeywordTok{aes}\NormalTok{(}\DataTypeTok{x =}\NormalTok{ x, }\DataTypeTok{y =}\NormalTok{ y)) }\OperatorTok{+}
\StringTok{  }\KeywordTok{geom_point}\NormalTok{() }\OperatorTok{+}
\StringTok{  }\KeywordTok{labs}\NormalTok{(}
    \DataTypeTok{x =} \StringTok{"Number of Cases"}\NormalTok{,}
    \DataTypeTok{y =} \StringTok{"Delivery Time"}
\NormalTok{  )}
\end{Highlighting}
\end{Shaded}

\begin{figure}[H]

{\centering \includegraphics[width=0.7\linewidth]{regression-analysis_files/figure-latex/delivery-1} 

}

\caption{The Delivery Time Data}\label{fig:delivery}
\end{figure}

Given data \((x_1, Y_1), \ldots, (x_n, Y_n)\), we try to fit linear
model

\[Y_i = \beta_0 + \beta_1 x_i + \epsilon_i\]

Here \(\epsilon_i\) is a error term, which is a random variable.

\[\epsilon \stackrel{iid}{\sim} (0, \sigma^2)\]

It gives the problem of estimating three parameters
\((\beta_0, \beta_1, \sigma^2)\). Before estimating these, we set some
assumptions.

\begin{enumerate}
\def\labelenumi{\arabic{enumi}.}
\tightlist
\item
  linear relationship
\item
  \(\epsilon_i\)s are independent
\item
  \(\epsilon_i\)s are identically destributed, i.e. \emph{constant
  variance}
\item
  In some setting, \(\epsilon_i \sim N\)
\end{enumerate}

\section{Least Squares Estimation}\label{least-squares-estimation}

\begin{figure}[H]

{\centering \includegraphics[width=0.7\linewidth]{regression-analysis_files/figure-latex/lsefig-1} 

}

\caption{Idea of the least square estimation}\label{fig:lsefig}
\end{figure}

We try to find \(\beta_0\) and \(\beta_1\) that minimize the sum of
squares of the vertical distances, i.e.

\begin{equation}
  (\beta_0, \beta_1) = \arg\min \sum_{i = 1}^n (Y_i - \beta_0 - \beta_1 x_i)^2
  \label{eq:ssq}
\end{equation}

\subsection{Normal equations}\label{normal-equations}

Denote that Equation \eqref{eq:ssq} is quadratic. Then we can find its
minimum by find the zero point of the first derivative. Set

\[Q(\beta_0, \beta_1) := \sum_{i = 1}^n (Y_i - \beta_0 - \beta_1 x_i)^2\]

Then we have

\begin{equation}
  \frac{\partial Q}{\partial \beta_0} = -2 \sum_{i = 1}^n(Y_i - \beta_0 - \beta_1 x_i) = 0
  \label{eq:normbeta0}
\end{equation}

and

\begin{equation}
  \frac{\partial Q}{\partial \beta_1} = -2 \sum_{i = 1}^n (Y_i - \beta_0 - \beta_1 x_i)x_i = 0
  \label{eq:normbeta1}
\end{equation}

From Equation \eqref{eq:normbeta0},

\[\sum_{i = 1}^n Y_i - n \hat\beta_0 - \hat\beta_1 \sum_{i = 1}^n x_i = 0\]

Thus,

\[\hat\beta_0 = \overline{Y} - \hat\beta_1 \overline{x}\]

Equation \eqref{eq:normbeta1} gives

\[\sum_{i = 1}^n x_i (Y_i - \overline{Y} + \hat\beta_1\overline{x} - \hat\beta_1 x_i) = \sum_{i = 1}^n x_i(Y_i - \overline{Y}) - \hat\beta_1\sum_{i = 1}^n x_i (x_i - \overline{x}) = 0\]

Thus,

\[\hat\beta_1 = \frac{\sum\limits_{i = 1}^nx_i(Y_i - \overline{Y})}{\sum\limits_{i = 1}^n x_i (x_i - \overline{x})}\]

\BeginKnitrBlock{remark}
\iffalse{} {Remark. } \fi{}\[\hat\beta_1 = \frac{S_{XY}}{S_{XX}}\]

where \(S_{XX} := \sum\limits_{i = 1}^n (x_i - \overline{x})^2\) and
\(S_{XY} := \sum\limits_{i = 1}^n (x_i - \overline{x})(Y_i - \overline{Y})\)
\EndKnitrBlock{remark}

\BeginKnitrBlock{proof}
\iffalse{} {Proof. } \fi{}Note that
\(\overline{x}^2 = \frac{1}{n^2}\bigg(\sum\limits_{i = 1}^n x_i\bigg)^2\).
Then we have

\begin{equation}
  \begin{split}
    S_{XX} & = \sum_{i = 1}^n (x_i - \overline{x})^2 \\
    & = \sum_{i = 1}^n x_i^2 - 2\sum_{i = 1}^n x_i \overline{x} + \sum_{i = 1}^n\overline{x}^2 \\
    & = \sum_{i = 1}^n x_i^2 - \frac{2}{n}\bigg(\sum\limits_{i = 1}^n x_i\bigg)^2 + \frac{1}{n}\bigg(\sum\limits_{i = 1}^n x_i\bigg)^2 \\
    & = \sum_{i = 1}^n x_i^2 - \frac{1}{n}\bigg(\sum\limits_{i = 1}^n x_i\bigg)^2
  \end{split}
  \label{eq:sxx}
\end{equation}

It follows that

\begin{equation*}
  \begin{split}
    \hat\beta_1 & = \frac{\sum x_i(Y_i - \overline{Y})}{\sum x_i (x_i - \overline{x})} \\
    & = \frac{\sum x_i (Y_i - \overline{Y}) - \overline{x}\sum (Y_i - \overline{Y})}{\sum x_i^2 - \frac{1}{n} (\sum x_i)^2} \qquad \because \sum (Y_i - \overline{Y}) = 0 \\
    & = \frac{\sum (x_i - \overline{x})(Y_i - \overline{Y})}{\sum x_i^2 - \frac{1}{n} (\sum x_i)^2} \\
    & = \frac{S_{XY}}{S_{XX}}
  \end{split}
\end{equation*}
\EndKnitrBlock{proof}

\begin{Shaded}
\begin{Highlighting}[]
\KeywordTok{lm}\NormalTok{(y }\OperatorTok{~}\StringTok{ }\NormalTok{x, }\DataTypeTok{data =}\NormalTok{ delv)}
\end{Highlighting}
\end{Shaded}

\begin{verbatim}

Call:
lm(formula = y ~ x, data = delv)

Coefficients:
(Intercept)            x  
       3.32         2.18  
\end{verbatim}

\subsection{Prediction and Mean
response}\label{prediction-and-mean-response}

\begin{quote}
``Essentially, all models are wrong, but some are useful.''

---George Box
\end{quote}

Recall that we have assumed the \textbf{linear assumption} between the
predictor and the response variables, i.e.~the true model. Estimating
\(\beta_0\) and \(\beta_1\) is same as estimating the \emph{assumed true
model}.

\BeginKnitrBlock{definition}[Mean response]
\protect\hypertarget{def:eyx}{}{\label{def:eyx} \iffalse (Mean response)
\fi{} }\[E(Y \mid X = x) = \beta_0 + \beta_1 x\]
\EndKnitrBlock{definition}

We can estimate this mean resonse by

\begin{equation}
  \widehat{E(Y \mid x)} = \hat\beta_0 + \hat\beta_1 x
  \label{eq:meanres}
\end{equation}

However, in practice, the model might not be true, which is included in
\(\epsilon\) term.

\[Y_i = \beta_0 + \beta_1 x_i + \epsilon_i\]

Our real problem is predicting individual \(Y\), not the mean. The
\emph{prediction} of response can be done by

\begin{equation}
  \hat{Y_i}  = \hat\beta_0 + \hat\beta_1 x_i
  \label{eq:indpred}
\end{equation}

Observe that the values of Equations \eqref{eq:meanres} and
\eqref{eq:indpred} are same. However, due to the \textbf{error term in the
prediction}, it has larger standard error.

\subsection{Properties of LSE}\label{lseprop}

Parameters \(\beta_0\) and \(\beta_1\) have some properties related to
the expectation and variance. We can notice that these lse's are
\textbf{unbiased linear estimator}. In fact, these are the \emph{best
unbiased linear estimator}. This will be covered in the Gauss-Markov
theorem.

\BeginKnitrBlock{lemma}
\protect\hypertarget{lem:sxy}{}{\label{lem:sxy}
}\[S_{XX} = \sum_{i = 1}^n x_i^2 - \frac{1}{n}\bigg(\sum\limits_{i = 1}^n x_i\bigg)^2 = \sum_{i = 1}^n x_i(x_i - \overline{x})\]

\[S_{XY} = \sum_{i = 1}^n x_i Y_i - \frac{1}{n}\bigg(\sum_{i = 1}^n x_i\bigg)\bigg(\sum_{i = 1}^n Y_i\bigg) = \sum_{i = 1}^n Y_i(x_i - \overline{x})\]
\EndKnitrBlock{lemma}

\BeginKnitrBlock{proof}
\iffalse{} {Proof. } \fi{}We already proven the first part of
\(S_{XX}\). See the Equation \eqref{eq:sxx}. The second part is tivial.
Since \(\sum (x_i - \overline{x}) = 0\),

\[S_{XX} = \sum_{i = 1}^n (x_i - \overline{x})^2 = \sum_{i = 1}^n (x_i - \overline{x})x_i\]

For the first part of \(S_{XY}\),

\begin{equation*}
  \begin{split}
    S_{XY} & = \sum_{i = 1}^n (x_i - \overline{x})(Y_i - \overline{Y}) \\
    & = \sum_{i = 1}^n x_i Y_i - \overline{x} \sum_{i = 1}^n Y_i - \overline{Y} \sum_{i = 1}^n x_i + n \overline{x} \overline{Y} \\
    & = \sum_{i = 1}^n x_i Y_i - \frac{1}{n}\bigg(\sum_{i = 1}^n x_i\bigg)\bigg(\sum_{i = 1}^n Y_i\bigg)
  \end{split}
\end{equation*}

Second part of \(S_{XY}\) also can be proven from the definition.

\begin{equation*}
  \begin{split}
    S_{XY} & = \sum_{i = 1}^n (x_i - \overline{x})(Y_i - \overline{Y}) \\
    & = \sum_{i = 1}^n Y_i (x_i - \overline{x}) - \overline{Y} \sum_{i = 1}^n (x_i - \overline{x}) \\
    & = \sum_{i = 1}^n Y_i (x_i - \overline{x}) \qquad \because \sum_{i = 1}^n (x_i - \overline{x}) = 0
  \end{split}
\end{equation*}
\EndKnitrBlock{proof}

\BeginKnitrBlock{lemma}[Linearity]
\protect\hypertarget{lem:linbet}{}{\label{lem:linbet} \iffalse (Linearity)
\fi{} }Each coefficient is a linear estimator.

\[\hat\beta_1 = \sum_{i = 1}^n\frac{(x_i - \overline{x})}{S_{XX}}Y_i\]

\[\hat\beta_0 = \sum_{i = 1}^n \bigg( \frac{1}{n} - \frac{(x_i - \overline{x})}{S_{XX}} \bigg) Y_i\]
\EndKnitrBlock{lemma}

\BeginKnitrBlock{proof}
\iffalse{} {Proof. } \fi{}From lemma \ref{lem:sxy},

\begin{equation*}
  \begin{split}
    \hat\beta_1 & = \frac{S_{XY}}{S_{XX}} \\
    & = \frac{1}{S_{XX}}\sum_{i = 1}^n (x_i - \overline{x}) Y_i
  \end{split}
\end{equation*}

It gives that

\begin{equation*}
  \begin{split}
    \hat\beta_0 & = \overline{Y} - \hat\beta_1 \overline{x} \\
    & = \frac{1}{n}\sum_{i = 1}^n Y_i - \overline{x} \sum_{i = 1}^n\frac{(x_i - \overline{x})}{S_{XX}}Y_i \\
    & = \sum_{i = 1}^n\bigg(\frac{1}{n} - \frac{(x_i - \overline{x})\overline{x}}{S_{XX}} \bigg)Y_i
  \end{split}
\end{equation*}
\EndKnitrBlock{proof}

\BeginKnitrBlock{proposition}[Unbiasedness]
\protect\hypertarget{prp:ue}{}{\label{prp:ue} \iffalse (Unbiasedness) \fi{}
}Both coefficients are unbiased.

\(\text{(a)}\: E\hat\beta_1 = \beta_1\)

\(\text{(b)}\: E\hat\beta_0 = \beta_0\)
\EndKnitrBlock{proposition}

From the model,
\(Y_1, \ldots, Y_n \stackrel{indep}{\sim} (\beta_0 + \beta_1 x_i, \sigma^2)\).

\BeginKnitrBlock{proof}
\iffalse{} {Proof. } \fi{}From lemma \ref{lem:sxy},

\begin{equation*}
  \begin{split}
    E\hat\beta_1 & = \sum_{i = 1}^n \bigg[ \frac{(x_i - \overline{x})}{S_{XX}} E(Y_i) \bigg] \\
    & = \sum_{i = 1}^n \frac{(x_i - \overline{x})}{S_{XX}}(\beta_0 + \beta_1 x_i) \\
    & = \frac{\beta_1 \sum (x_i - \overline{x})x_i}{\sum (x_i - \overline{x})x_i} \qquad \because \sum (x_i - \overline{x}) = 0 \\
    & = \beta_1
  \end{split}
\end{equation*}

It follows that

\begin{equation*}
  \begin{split}
    E\hat\beta_0 & = E(\overline{Y} - \hat\beta_1 \overline{x}) \\
    & = E(\overline{Y}) - \overline{x}E(\hat\beta_1) \\
    & = E(\beta_0 + \beta_1 \overline{x} + \overline{\epsilon}) - \beta_1 \overline{x} \\
    & = \beta_0 + \beta_1 \overline{x} - \beta_1 \overline{x} \\
    & = \beta_0
  \end{split}
\end{equation*}
\EndKnitrBlock{proof}

\BeginKnitrBlock{proposition}[Variances]
\protect\hypertarget{prp:vb}{}{\label{prp:vb} \iffalse (Variances) \fi{}
}Variances and covariance of coefficients

\(\text{(a)}\: Var\hat\beta_1 = \frac{\sigma^2}{S_{XX}}\)

\(\text{(b)}\: Var\hat\beta_0 = \bigg( \frac{1}{n} + \frac{\overline{x}^2}{S_{XX}} \bigg)\sigma^2\)

\(\text{(c)}\: Cov(\hat\beta_0, \hat\beta_1) = - \frac{\overline{x}}{S_{XX}} \sigma^2\)
\EndKnitrBlock{proposition}

\BeginKnitrBlock{proof}
\iffalse{} {Proof. } \fi{}Proving is just arithmetic.

\begin{enumerate}
\def\labelenumi{(\alph{enumi})}
\item
\end{enumerate}

\begin{equation*}
  \begin{split}
    Var\hat\beta_1 & = \frac{1}{S_{XX}^2}\sum_{i = 1}^n \bigg[ (x_i - \overline{x})^2 Var(Y_i) \bigg] + \frac{1}{S_{XX}^2} \sum_{j \neq k}^n \bigg[ (x_j - \overline{x})(x_k - \overline{x}) Cov(Y_j, Y_k) \bigg] \\
    & = \frac{\sigma^2}{S_{XX}} \qquad \because Cov(Y_j, Y_k) = 0 \: \text{if} \: j \neq k
  \end{split}
\end{equation*}

\begin{enumerate}
\def\labelenumi{(\alph{enumi})}
\setcounter{enumi}{1}
\item
\end{enumerate}

\begin{equation*}
  \begin{split}
    Var\hat\beta_0 & = \sum_{i = 1}^n \bigg( \frac{1}{n} - \frac{(x_i - \overline{x})\overline{x}}{S_{XX}} \bigg)^2Var(Y_i) + \sum_{j \neq k} \bigg( \frac{1}{n} - \frac{(x_j - \overline{x})\overline{x}}{S_{XX}} \bigg)\bigg( \frac{1}{n} - \frac{(x_k - \overline{x})\overline{x}}{S_{XX}} \bigg) Cov(Y_j, Y_k) \\
    & = \frac{\sigma^2}{n} - 2 \sigma^2 \frac{\overline{x}}{S_{XX}} \sum_{i = 1}^n (x_i - \overline{x}) + \frac{\sigma^2 \overline{x}^2 \sum (x_i - \overline{x})^2}{S_{XX}^2} \\
    & = \bigg( \frac{1}{n} + \frac{\overline{x}^2}{S_{XX}} \bigg) \sigma^2 \qquad \because \sum (x_i - \overline{x}) = 0
  \end{split}
\end{equation*}

\begin{enumerate}
\def\labelenumi{(\alph{enumi})}
\setcounter{enumi}{2}
\item
\end{enumerate}

\begin{equation*}
  \begin{split}
    Cov(\hat\beta_0, \hat\beta_1) & = Cov(\overline{Y} - \hat\beta_1 \overline{x}, \hat\beta_1) \\
    & = - \overline{x} Var\hat\beta_1 \\
    & = - \frac{\overline{x}}{S_{XX}} \sigma^2
  \end{split}
\end{equation*}
\EndKnitrBlock{proof}

\subsection{Gauss-Markov Theorem}\label{gauss-markov-theorem}

Chapter \ref{lseprop} shows that the \(\beta_0^{LSE}\) and
\(\beta_1^{LSE}\) are the \textbf{linear unbiased estimators}. Are these
good? Good compared to \emph{what estimators}? Here we consider
\emph{linear unbiased estimator}. If variances in the proposition
\ref{prp:vb} are lower than any parameters in this parameter family,
\(\beta_0^{LSE}\) and \(\beta_1^{LSE}\) are the \textbf{best linear
unbiased estimators}.

\BeginKnitrBlock{theorem}[Gauss Markov Theorem]
\protect\hypertarget{thm:gmt}{}{\label{thm:gmt} \iffalse (Gauss Markov
Theorem) \fi{} }\(\hat\beta_0\) and \(\hat\beta_1\) are BLUE, i.e.~the
best linear unbiased estimator.

\[Var(\hat\beta_0) \le Var\Big( \sum_{i = 1}^n a_i Y_i \Big) \: \forall a_i \in \mathbb{R} \: \text{s.t.} \: E\Big( \sum_{i = 1}^n a_i Y_i \Big) = \beta_0\]

\[Var(\hat\beta_1) \le Var\Big( \sum_{i = 1}^n b_i Y_i \Big) \: \forall b_i \in \mathbb{R} \: \text{s.t.} \: E\Big( \sum_{i = 1}^n b_i Y_i \Big) = \beta_1\]
\EndKnitrBlock{theorem}

\BeginKnitrBlock{proof}[Bestness of beta1]
\iffalse{} {Proof (Bestness of beta1). } \fi{}Consider
\(\Theta := \bigg\{ \sum\limits_{i = 1}^n b_i Y_i \in \mathbb{R} : E\Big( \sum\limits_{i = 1}^n b_i Y_i \Big) = \beta_1 \bigg\}\).

Claim: \(Var( \sum b_i Y_i) - Var(\hat\beta_1) \ge 0\)

Let \(\sum b_i Y_i \in \Theta\). Then \(E(\sum b_i Y_i) = \beta_1\).

Since \(E(Y_i) = \beta_0 + \beta_1 x_i\),

\[\beta_0 \sum b_i + \beta_1 \sum b_i x_i = \beta_1\]

It gives

\begin{equation} \label{eq:ule}
  \begin{cases}
    \sum b_i = 0 \\
    \sum b_i x_i = 1
  \end{cases}
\end{equation}

Then

\begin{equation*}
  \begin{split}
    0 \le Var\Big(\sum b_i Y_i - \hat\beta_1\Big) & = Var\Big( \sum b_i Y_i - \sum \frac{(x_i - \bar{x})}{S_{XX}} Y_i \Big) \\
    & \stackrel{indep}{=} \sum \bigg( b_i - \frac{(x_i - \bar{x})}{S_{XX}} \bigg)^2 \sigma^2 \\
    & = \sum \bigg( b_i^2 - \frac{2b_i (x_i - \bar{x})}{S_{XX}} + \frac{(x_i - \bar{x})^2}{S_{XX}^2} \bigg) \sigma^2 \\
    & = \sum b_i^2 \sigma^2 - \frac{2 \sigma^2}{S_{XX}} \sum b_i x_i + \frac{2 \bar{x} \sigma^2}{S_{XX}} \sum b_i + \sigma^2 \frac{\sum (x_i - \bar{x})^2}{S_{XX}^2} \\
    & = \sum b_i^2 \sigma^2 - \frac{\sigma^2}{S_{XX}} \qquad \because \eqref{eq:ule} \:\text{and}\: S_{XX} = \sum (x_i - \bar{x})^2 \\
    & = Var(\sum b_i Y_i) - Var(\hat\beta_1)
  \end{split}
\end{equation*}

Hence,

\[Var(\sum b_i Y_i) \ge Var(\hat\beta_1)\]
\EndKnitrBlock{proof}

\BeginKnitrBlock{proof}[Bestness of beta0]
\iffalse{} {Proof (Bestness of beta0). } \fi{}Consider
\(\Theta := \bigg\{ \sum\limits_{i = 1}^n a_i Y_i \in \mathbb{R} : E\Big( \sum\limits_{i = 1}^n a_i Y_i \Big) = \beta_0 \bigg\}\).

Claim: \(Var( \sum a_i Y_i) - Var(\hat\beta_0) \ge 0\)

Let \(\sum a_i Y_i \in \Theta\). Then \(E(\sum a_i Y_i) = \beta_0\).

Since \(E(Y_i) = \beta_0 + \beta_1 x_i\),

\[\beta_0 \sum a_i + \beta_1 \sum a_i x_i = \beta_0\]

It gives

\begin{equation} \label{eq:ule0}
  \begin{cases}
    \sum a_i = 1 \\
    \sum a_i x_i = 0
  \end{cases}
\end{equation}

Then

\begin{equation*}
  \begin{split}
    0 \le Var\Big(\sum a_iY_i - \hat\beta_0 \Big) & = Var\bigg[\sum a_iY_i - \sum\Big( \frac{1}{n} - \frac{(x_k - \bar{x})\bar{x}}{S_{XX}} \Big) Y_k \bigg] \\
    & = \sum \bigg(a_i - \frac{1}{n} +  \frac{(x_i - \bar{x})\bar{x}}{S_{XX}} \bigg)^2\sigma^2 \\
    & = \sum \bigg[ a_i^2 - 2a_i\Big( \frac{1}{n} - \frac{(x_i - \bar{x})\bar{x}}{S_{XX}} \Big) + \Big( \frac{1}{n} - \frac{(x_i - \bar{x})\bar{x}}{S_{XX}} \Big)^2 \bigg]\sigma^2 \\
        & = \sum a_i^2\sigma^2 -\frac{2\sigma^2}{n}\sum a_i + \frac{2\bar{x}\sigma^2\sum a_ix_i}{S_{XX}} - \frac{2\bar{x}^2\sigma^2\sum a_i}{S_{XX}} \\
        & \qquad + \sigma^2\bigg( \frac{1}{n} - \frac{2\bar{x}}{nS_{XX}} \sum(x_i - \bar{x}) + \frac{\bar{x}^2\sum(x_i - \bar{x})^2}{S_{XX}^2} \bigg) \\
        & = \sum a_i^2\sigma^2 -\frac{2\sigma^2}{n} - \frac{2\bar{x}^2\sigma^2}{S_{XX}} \qquad \because \eqref{eq:ule0} \\
        & \qquad + \bigg(\frac{1}{n} + \frac{\bar{x}^2}{S_{XX}} \bigg)\sigma^2 \qquad \because \sum(x_i - \bar{x}) = 0 \: \text{and} \: S_{XX} := \sum (x_i - \bar{x})^2 \\
        & = \sum a_i^2\sigma^2 - \bigg( \frac{1}{n} + \frac{\bar{x}^2}{S_{XX}} \bigg)\sigma^2 \\
        & = Var\Big( \sum a_i Y_i \Big) - Var\hat\beta_0
  \end{split}
\end{equation*}

Hence,

\[Var(\sum a_i Y_i) \ge Var(\hat\beta_0)\]
\EndKnitrBlock{proof}

\BeginKnitrBlock{example}
\protect\hypertarget{exm:usingnormal}{}{\label{exm:usingnormal} }Show that
\(\sum (Y_i - \hat{Y_i}) = 0\), \(\sum x_i (Y_i - \hat{Y_i}) = 0\), and
\(\sum \hat{Y_i} (Y_i - \hat{Y_i}) = 0\).
\EndKnitrBlock{example}

\BeginKnitrBlock{solution}
\iffalse{} {Solution. } \fi{}Consider the two normal equations
\eqref{eq:normbeta0} and \eqref{eq:normbeta1}. Note that
\(\hat{Y_i} = \hat\beta_0 + \hat\beta_1 x_i\).

From the Equation \eqref{eq:normbeta0}, we have
\(\sum (Y_i - \hat{Y_i}) = 0\).

From the Equation \eqref{eq:normbeta1}, we have
\(\sum x_i (Y_i - \hat{Y_i}) = 0\).

It follows that

\begin{equation*}
  \begin{split}
    \sum \hat{Y_i} (Y_i - \hat{Y_i}) & = \sum (\hat\beta_0 + \hat\beta_1 x_i) (Y_i - \hat{Y_i}) \\
    & = \hat\beta_0 \sum (Y_i - \hat{Y_i}) + \hat\beta_1 \sum x_i (Y_i - \hat{Y_i}) \\
    & = 0
  \end{split}
\end{equation*}
\EndKnitrBlock{solution}

\subsection{\texorpdfstring{Estimation of
\(\sigma^2\)}{Estimation of \textbackslash{}sigma\^{}2}}\label{estimation-of-sigma2}

There is the last parameter, \(\sigma^2 = Var(Y_i)\). In the \emph{least
squares estimation literary}, we estimate \(\sigma^2\) by

\begin{equation} \label{eq:siglse}
  \hat\sigma^2 = \frac{1}{n - 2} \sum_{i = 1}^n (Y_i - \hat\beta_0 - \hat\beta_1 x_i)^2
\end{equation}

Why \(n - 2\)? This makes the estimator unbiased.

\BeginKnitrBlock{proposition}[Unbiasedness]
\protect\hypertarget{prp:sigex}{}{\label{prp:sigex} \iffalse (Unbiasedness)
\fi{} }\[E(\hat\sigma^2) = \sigma^2\]
\EndKnitrBlock{proposition}

\BeginKnitrBlock{proof}
\iffalse{} {Proof. } \fi{}Note that

\[(Y_i - \hat\beta_0 - \hat\beta_1 x_i) = (Y_i - \overline{Y}) - \hat\beta_1(x_i - \overline{x})\]

Then

\begin{equation*}
  \begin{split}
    E(\hat\sigma^2) & = \frac{1}{n - 2} E \bigg[ \sum (Y_i - \hat\beta_0 - \hat\beta_1 x_i)^2 \bigg] \\
    & = \frac{1}{n - 2} E \bigg[ \sum (Y_i - \overline{Y})^2 + \hat\beta_1^2 \sum (x_i - \overline{x})^2 -2\hat\beta_1 \sum (Y_i - \overline{Y})(x_i - \overline{x}) \bigg] \\
    & = \frac{1}{n - 2} E ( S_{YY} + \hat\beta_1^2 S_{XX} - 2 \hat\beta_1 S_{XY}) \\
    & = \frac{1}{n - 2} E ( S_{YY} - \hat\beta_1^2 S_{XX}) \qquad \because S_{XY} = \hat\beta_1 S_{XX} \\
    & = \frac{1}{n - 2} \Big(  \underset{(a)}{\underline{ES_{YY}}} - S_{XX} \underset{(b)}{\underline{E\hat\beta_1^2}} \Big)
  \end{split}
\end{equation*}

\begin{enumerate}
\def\labelenumi{(\alph{enumi})}
\item
\end{enumerate}

\begin{equation*}
  \begin{split}
    ES_{YY} & = E\Big[ \sum (Y_i - \overline{Y})^2 \Big] \\
    & = E \Big[ \sum \Big( (\beta_0 + \beta_1 x_i + \epsilon_i) - (\beta_0 + \beta_1 \overline{x} + \overline{\epsilon}) \Big)^2 \Big] \\
    & = E \Big[ \sum \Big( \beta_1 (x_i - \overline{x}) + (\epsilon_i - \overline{\epsilon}) \Big)^2 \Big] \\
    & = \beta_1^2 S_{XX} + E\Big( \sum (\epsilon_i - \overline{\epsilon})^2 \Big) + 2\beta_1 \sum (x_i - \overline{x}) E(\epsilon_i - \overline{\epsilon}) \\
    & = \beta_1^2 S_{XX} + E\Big( \sum (\epsilon_i - \overline{\epsilon})^2 \Big)
  \end{split}
\end{equation*}

Since \(E(\bar\epsilon) = 0\) and
\(Var(\bar\epsilon) = \frac{\sigma^2}{n}\),

\begin{equation*}
  \begin{split}
    E\Big( \sum (\epsilon_i - \overline{\epsilon})^2 \Big) & = E \Big( \sum (\epsilon_i^2 + \bar\epsilon^2 - 2\epsilon_i \bar\epsilon) \Big) \\
    & = \sum E(\epsilon_i^2) - n E(\bar\epsilon^2) \qquad \because \sum \epsilon = n \bar\epsilon \\
    & = \sum (Var(\epsilon_i) + E(\epsilon_i)^2) - n(Var(\bar\epsilon) + E(\bar\epsilon)^2) \\
    & = n\sigma^2 - \sigma^2 \\
    & = (n - 1)\sigma^2
  \end{split}
\end{equation*}

Thus,

\[ES_{YY} = \beta_1^2 S_{XX} + (n - 1)\sigma^2\]

\begin{enumerate}
\def\labelenumi{(\alph{enumi})}
\setcounter{enumi}{1}
\item
\end{enumerate}

\begin{equation*}
  \begin{split}
    E\hat\beta_1^2 & = Var\hat\beta_1 + E(\hat\beta_1)^2 \\
    & = \frac{\sigma^2}{S_{XX}} + \beta_1^2
  \end{split}
\end{equation*}

It follows that

\begin{equation*}
  \begin{split}
    E(\hat\sigma^2) & = \frac{1}{n - 2} \Big(  \underset{(a)}{\underline{ES_{YY}}} - S_{YY} \underset{(b)}{\underline{E\hat\beta_1^2}} \Big) \\
    & = \frac{1}{n - 2} \bigg( \Big(\beta_1^2 S_{XX} + (n - 1)\sigma^2 \Big) - S_{XX}\Big(\frac{\sigma^2}{S_{XX}} + \beta_1^2 \Big) \bigg) \\
    & = \frac{1}{n - 2}((n - 2)\sigma^2) \\
    & = \sigma^2
  \end{split}
\end{equation*}
\EndKnitrBlock{proof}

\section{Maximum Likelihood
Estimation}\label{maximum-likelihood-estimation}

In this section, we add an assumption to an random errors
\(\epsilon_i\).

\[\epsilon_i \stackrel{iid}{\sim} N(0, \sigma^2)\]

\BeginKnitrBlock{example}[Gaussian Likelihood]
\protect\hypertarget{exm:gmle}{}{\label{exm:gmle} \iffalse (Gaussian
Likelihood) \fi{} }Note that
\(Y_i \stackrel{indep}{\sim} N(\beta_0 + \beta_1 x_i, \sigma^2)\). Then
the likelihood function is

\[L(\beta_0, \beta_1, \sigma^2) = \prod_{i = 1}^n\bigg( \frac{1}{\sqrt{2\pi\sigma^2}} \exp \bigg(- \frac{(Y_i - \beta_0 - \beta_1 x_i)^2}{2 \sigma^2} \bigg) \bigg)\]

and so the log-likelihood function can be computed as

\[l(\beta_0, \beta_1, \sigma^2) = -\frac{n}{2}\ln(2\pi\sigma^2) - \frac{1}{2\sigma^2}\sum_{i = 1}^n(Y_i - \beta_0 - \beta_1 x_i)^2\]
\EndKnitrBlock{example}

\subsection{Likelihood equations}\label{likelihood-equations}

\BeginKnitrBlock{definition}[Maximum Likelihood Estimator]
\protect\hypertarget{def:mledef}{}{\label{def:mledef} \iffalse (Maximum
Likelihood Estimator) \fi{}
}\[(\hat\beta_0^{MLE}, \hat\beta_1^{MLE}, \hat\sigma^{2MLE}) := \arg\sup L(\beta_0, \beta_1, \sigma^2)\]
\EndKnitrBlock{definition}

Since \(l(\cdot) = \ln L(\cdot)\) is monotone,

\BeginKnitrBlock{remark}
\iffalse{} {Remark. }
\fi{}\[(\hat\beta_0^{MLE}, \hat\beta_1^{MLE}, \hat\sigma^{2MLE}) = \arg\sup l(\beta_0, \beta_1, \sigma^2)\]
\EndKnitrBlock{remark}

We can find the maximum of this \emph{quadratic} function by making
first derivative.

\begin{equation}
  \frac{\partial l}{\partial \beta_0} = \frac{1}{\sigma^2} \sum_{i = 1}^n (Y_i - \beta_0 - \beta_1 x_i) = 0
  \label{eq:mlbeta0}
\end{equation}

\begin{equation}
  \frac{\partial l}{\partial \beta_1} = \frac{1}{\sigma^2} \sum_{i = 1}^n x_i (Y_i - \beta_0 - \beta_1 x_i) = 0
  \label{eq:mlbeta1}
\end{equation}

\begin{equation}
  \frac{\partial l}{\partial \sigma^2} = - \frac{n}{2\sigma^2} + \frac{1}{2\sigma^4} \sum_{i = 1}^n (Y_i - \beta_0 - \beta_1 x_i)^2 = 0
  \label{eq:mlsig}
\end{equation}

Denote that Equations \eqref{eq:mlbeta0} and \eqref{eq:mlbeta1} given
\(\hat\sigma^2\) are equivalent to the normal equations. Thus,

\[\hat\beta_0^{MLE} = \hat\beta_0^{LSE}, \quad \hat\beta_1^{MLE} = \hat\beta_1^{LSE}\]

From Equation \eqref{eq:mlsig},

\[\hat\sigma^{2MLE} = \frac{1}{n}\sum_{i = 1}^n(Y_i - \beta_0 - \beta_1 x_i)^2 = \frac{n - 2}{n} \hat\sigma^{2LSE}\]

Recall that \(\hat\sigma^{2LSE}\) is an unbiased, i.e.~this \emph{MLE is
not an unbiased estimator}. Since
\(\hat\sigma^{2MLE} \approx \hat\sigma^{2LSE}\) for large \(n\),
howerver, it is \emph{asymptotically unbiased}.

\BeginKnitrBlock{theorem}[Rao-Cramer Lower Bound, univariate case]
\protect\hypertarget{thm:rclb}{}{\label{thm:rclb} \iffalse (Rao-Cramer Lower
Bound, univariate case) \fi{} }Let
\(X_1, \ldots, X_n \stackrel{iid}{\sim} f(x ; \theta)\). If
\(\hat\theta\) is an unbiased estimator of \(\theta\),

\[Var(\hat\theta) \ge \frac{1}{I_n(\theta)}\]

where
\(I_n(\theta) = -E\bigg(\frac{\partial^2 l(\theta)}{\partial \theta^2} \bigg)\)
\EndKnitrBlock{theorem}

To apply this theorem \ref{thm:rclb} in the simple linear regression
setting, i.e. \((\beta_0, \beta_1)\), we need to look at the
\emph{bivariate case}.

\BeginKnitrBlock{theorem}[Rao-Cramer Lower Bound, bivariate case]
\protect\hypertarget{thm:rclb2}{}{\label{thm:rclb2} \iffalse (Rao-Cramer
Lower Bound, bivariate case) \fi{} }Let
\(X_1, \ldots, X_n \stackrel{iid}{\sim} f(x ; \theta1, \theta_2)\) and
let \(\boldsymbol{\theta} = (\theta_1, \theta_2)^T\). If each
\(\hat\theta_1\), \(\hat\theta_2\) is an unbiased estimator of
\(\theta_1\) and \(\theta_2\), then

\[
Var(\boldsymbol{\theta}) := \begin{bmatrix}
Var(\hat\theta_1) & Cov(\hat\theta_1, \hat\theta_2) \\
Cov(\hat\theta_1, \hat\theta_2) & Var(\hat\theta_2)
\end{bmatrix} \ge I_n^{-1}(\theta_1, \theta_2)
\]

where

\[
I_n(\theta_1, \theta_2) = - \begin{bmatrix}
  E\bigg( \frac{\partial^2 l(\theta_1, \theta_2)}{\partial \theta_1^2} \bigg) & E\bigg( \frac{\partial^2 l(\theta_1, \theta_2)}{\partial \theta_1 \partial \theta_2} \bigg) \\
  E\bigg( \frac{\partial^2 l(\theta_1, \theta_2)}{\partial \theta_1 \partial \theta_2} \bigg) & E\bigg( \frac{\partial^2 l(\theta_1, \theta_2)}{\partial \theta_2^2} \bigg)
\end{bmatrix}
\]
\EndKnitrBlock{theorem}

Assume that \(\sigma^2\) is \textbf{known}. From the Equations
\eqref{eq:mlbeta0} and \eqref{eq:mlbeta1},

\[
\begin{cases}
  \frac{\partial^2 l}{\partial \beta_0^2} = - \frac{n}{\sigma^2} \\
  \frac{\partial^2 l}{\partial \beta_1^2} = - \frac{\sum x_i^2}{\sigma^2} \\
  \frac{\partial^2 l}{\partial \beta_0 \partial \beta_1} = - \frac{\sum x_i}{\sigma^2}
\end{cases}
\]

Thus,

\[
I_n(\beta_0, \beta_1) = \begin{bmatrix}
  \frac{n}{\sigma^2} & \frac{\sum x_i}{\sigma^2} \\
  \frac{\sum x_i}{\sigma^2} & \frac{\sum x_i^2}{\sigma^2}
\end{bmatrix}
\]

Applying gaussian elimination,

\begin{equation*}
  \begin{split}
    \left[
    \begin{array}{cc|cc}
      \frac{n}{\sigma^2} & \frac{\sum x_i}{\sigma^2} & 1 & 0 \\
      \frac{\sum x_i}{\sigma^2} & \frac{\sum x_i^2}{\sigma^2} & 0 & 1
    \end{array}
    \right] & \leftrightarrow \left[
    \begin{array}{cc|cc}
      \frac{n}{\sigma^2} & \frac{\sum x_i}{\sigma^2} & 1 & 0 \\
      \frac{\sum x_i}{\sigma^2}\Big(\frac{n}{\sum x_i} \Big) & \frac{\sum x_i^2}{\sigma^2}\Big(\frac{n}{\sum x_i} \Big) & 0 & \frac{1}{\overline{x}}
    \end{array}
    \right] \\
    & \leftrightarrow \left[
    \begin{array}{cc|cc}
      \frac{n}{\sigma^2} & \frac{\sum x_i}{\sigma^2} & 1 & 0 \\
      0 & \frac{\sum x_i^2 - \overline{x}\sum x_i}{\sigma^2\overline{x}} = \frac{S_{XX}}{\sigma^2\overline{x}} & -1 & \frac{1}{\overline{x}}
    \end{array}
    \right] \\
    & \leftrightarrow \left[
    \begin{array}{cc|cc}
      1 & \overline{x} & \frac{\sigma^2}{n} & 0 \\
      0 & 1 & -\frac{\overline{x}}{S_{XX}}\sigma^2 & \frac{\sigma^2}{S_{XX}}
    \end{array}
    \right] \\
    & \leftrightarrow \left[
    \begin{array}{cc|cc}
      1 & 0 & \bigg(\frac{1}{n} + \frac{\overline{x}^2}{S_{XX}} \bigg)\sigma^2 & -\frac{\overline{x}}{S_{XX}}\sigma^2 \\
      0 & 1 & -\frac{\overline{x}}{S_{XX}}\sigma^2 & \frac{\sigma^2}{S_{XX}}
    \end{array}
    \right]
  \end{split}
\end{equation*}

Hence,

\[
I_n^{-1}(\beta_0, \beta_1) = \begin{bmatrix}
  \bigg(\frac{1}{n} + \frac{\overline{x}^2}{S_{XX}} \bigg)\sigma^2 & -\frac{\overline{x}}{S_{XX}}\sigma^2 \\
  -\frac{\overline{x}}{S_{XX}}\sigma^2 & \frac{\sigma^2}{S_{XX}}
\end{bmatrix} = \begin{bmatrix}
  Var(\hat\beta_0) & Cov(\hat\beta_0, \hat\beta_1) \\
  Cov(\hat\beta_0, \hat\beta_1) & Var(\hat\beta_1)
\end{bmatrix}
\]

Since \(Var(\boldsymbol{\hat\beta}) - I^{-1} = 0\) is non-negative
definite, each
\(Var(\hat\beta_0) = \bigg(\frac{1}{n} + \frac{\overline{x}^2}{S_{XX}} \bigg)\sigma^2\)
and \(Var(\hat\beta_1) = \frac{\sigma^2}{S_{XX}}\) is a theoretical
bound.

\BeginKnitrBlock{remark}
\iffalse{} {Remark. } \fi{}This says that
\(\hat\beta_0^{LSE} = \hat\beta_0^{MLE}\) and
\(\hat\beta_1^{LSE} = \hat\beta_1^{MLE}\) have the smallest variance
among all unbiased estimator.
\EndKnitrBlock{remark}

This result is \emph{stronger than Gauss-Markov theorem} \ref{thm:gmt},
where the LSE has the smalleset variance among all \emph{linear
unbiased} estimators. It can be simply obtained from the
\emph{Lehmann-Scheffe Theorem}: If some unbiased estimator is a function
of complete sufficient statistic, then this estimator is the unique MVUE
\citep{Hogg:2018aa}.

\BeginKnitrBlock{remark}[Lehmann and Scheffe for regression coefficients]
\iffalse{} {Remark (Lehmann and Scheffe for regression coefficients). }
\fi{}\(u\Big(\sum Y_i, S_{XY} \Big)\) is CSS in this regression problem,
i.e.~known \(\sigma^2\).
\EndKnitrBlock{remark}

\BeginKnitrBlock{proof}
\iffalse{} {Proof. } \fi{}From the example \ref{exm:gmle},

\begin{equation*}
  \begin{split}
    L(\beta_0, \beta_1) & = (2\pi\sigma^2)^{-\frac{n}{2}}\exp\bigg[-\frac{1}{2\sigma^2} \sum(Y_i - \beta_0 - \beta_1 x_i)^2 \bigg] \\
    & = (2\pi\sigma^2)^{-\frac{n}{2}}\exp\bigg[-\frac{1}{2\sigma^2} \sum \Big(Y_i^2 - (\beta_0 + \beta_1 x_i)Y_i + (\beta_0 + \beta_1 x_i)^2 \Big) \bigg] \\
    & = (2\pi\sigma^2)^{-\frac{n}{2}}\exp\bigg[-\frac{1}{2\sigma^2} \Big( -\beta_0 \sum Y_i - \beta_1 \sum x_i Y_i \Big) \bigg] \exp\bigg[-\frac{1}{2\sigma^2} \Big( \sum Y_i^2 + (\beta_0 + \beta_1 x_i)^2 \Big) \bigg]
  \end{split}
\end{equation*}

By the Factorization theorem, both \(\sum Y_i\) and \(\sum x_i Y_i\) are
sufficient statistics. Since \(S_{XY}\) is one-to-one function of
\(\sum x_i Y_i\), it is also a sufficient statistic.

Denote that the normal distribution is in exponential family.

Hence, \((\sum Y_i, S_{XY})\) are CSS.
\EndKnitrBlock{proof}

\section{Residuals}\label{residuals}

\BeginKnitrBlock{definition}[Residuals]
\protect\hypertarget{def:res}{}{\label{def:res} \iffalse (Residuals) \fi{}
}\[e_i := Y_i - \hat{Y_i}\]
\EndKnitrBlock{definition}

\subsection{Prediction error}\label{prediction-error}

\begin{Shaded}
\begin{Highlighting}[]
\NormalTok{delv }\OperatorTok\StringTok{ }
\StringTok{  }\KeywordTok{mutate}\NormalTok{(}\DataTypeTok{yhat =} \KeywordTok{predict}\NormalTok{(}\KeywordTok{lm}\NormalTok{(y }\OperatorTok{~}\StringTok{ }\NormalTok{x))) }\OperatorTok\StringTok{ }
\StringTok{  }\KeywordTok{ggplot}\NormalTok{(}\KeywordTok{aes}\NormalTok{(}\DataTypeTok{x =}\NormalTok{ x, }\DataTypeTok{y =}\NormalTok{ y)) }\OperatorTok{+}
\StringTok{  }\KeywordTok{geom_smooth}\NormalTok{(}\DataTypeTok{method =} \StringTok{"lm"}\NormalTok{, }\DataTypeTok{se =} \OtherTok{FALSE}\NormalTok{) }\OperatorTok{+}
\StringTok{  }\KeywordTok{geom_point}\NormalTok{() }\OperatorTok{+}
\StringTok{  }\KeywordTok{geom_linerange}\NormalTok{(}\KeywordTok{aes}\NormalTok{(}\DataTypeTok{ymin =}\NormalTok{ y, }\DataTypeTok{ymax =}\NormalTok{ yhat), }\DataTypeTok{col =} \KeywordTok{I}\NormalTok{(}\StringTok{"red"}\NormalTok{), }\DataTypeTok{alpha =}\NormalTok{ .}\DecValTok{7}\NormalTok{) }\OperatorTok{+}
\StringTok{  }\KeywordTok{labs}\NormalTok{(}
    \DataTypeTok{x =} \StringTok{"Number of Cases"}\NormalTok{,}
    \DataTypeTok{y =} \StringTok{"Delivery Time"}
\NormalTok{  )}
\end{Highlighting}
\end{Shaded}

\begin{figure}[H]

{\centering \includegraphics[width=0.7\linewidth]{regression-analysis_files/figure-latex/regplot-1} 

}

\caption{Fit and residuals}\label{fig:regplot}
\end{figure}

See Figure \ref{fig:regplot}. Each red line is \(e_i\). As we can see,
\(e_i\) represents the difference between \emph{observed} response and
\emph{predicted} response. A large \(\lvert e_i \rvert\) indicates a
large prediction error. You can call this \(e_i\) for each \(Y_i\) by
\texttt{lm()\$residuals} or \texttt{residuals()}.

\begin{Shaded}
\begin{Highlighting}[]
\NormalTok{delv_fit <-}\StringTok{ }\KeywordTok{lm}\NormalTok{(y }\OperatorTok{~}\StringTok{ }\NormalTok{x, }\DataTypeTok{data =}\NormalTok{ delv)}
\NormalTok{delv_fit}\OperatorTok{$}\NormalTok{residuals}
\end{Highlighting}
\end{Shaded}

\begin{verbatim}
     1      2      3      4      5      6      7      8      9     10 
-1.874  1.651  2.181  2.855 -2.628 -0.444  0.327 -0.724 10.634  7.298 
    11     12     13     14     15     16     17     18     19     20 
 2.191 -4.082  1.475  3.372  1.094  3.918 -1.028  0.446 -0.349 -5.216 
    21     22     23     24     25 
-7.182 -7.581 -4.156 -0.900 -1.275 
\end{verbatim}

\(\sum e_i^2\), which has been minimized in the procedure of LSE, can be
used to see \emph{overall size of prediction errors}.

\BeginKnitrBlock{definition}[Residual Sum of Squares]
\protect\hypertarget{def:sse}{}{\label{def:sse} \iffalse (Residual Sum of
Squares) \fi{} }\[SSE := \sum_{i = 1}^n e_i^2\]
\EndKnitrBlock{definition}

\subsection{Residuals and the
variance}\label{residuals-and-the-variance}

\(e_i\) is a random quantity, which contains the information for
\(\epsilon_i\). \(\sum e_i^2\) can give information about
\(\sigma^2 = Var(\epsilon_i)\). For this, it is expected that \(e_i\)
and \(\epsilon_i\) have similar feature.

\BeginKnitrBlock{lemma}
\protect\hypertarget{lem:yandbet}{}{\label{lem:yandbet} }Covriance between Y
and each coefficient

\(\text{(a)}\: Cov(\hat\beta_0, Y_i) = \bigg( \frac{1}{n} - \frac{(x_i - \overline{x})\overline{x}}{S_{XX}} \bigg)\sigma^2\)

\(\text{(b)}\: Cov(\hat\beta_1, Y_i) = \frac{(x_i - \overline{x})}{S_{XX}}\sigma^2\)
\EndKnitrBlock{lemma}

\BeginKnitrBlock{proof}
\iffalse{} {Proof. } \fi{}(a)

\begin{equation*}
  \begin{split}
    Cov(\hat\beta_0, Y_i) & = Cov(\sum a_i Y_i, Y_i) \\
    & = \bigg( \frac{1}{n} - \frac{(x_i - \overline{x})\overline{x}}{S_{XX}} \bigg)\sigma^2
  \end{split}
\end{equation*}

\begin{enumerate}
\def\labelenumi{(\alph{enumi})}
\setcounter{enumi}{1}
\item
\end{enumerate}

\begin{equation*}
  \begin{split}
    Cov(\hat\beta_1, Y_i) & = Cov(\sum b_i Y_i, Y_i) \\
    & = \frac{(x_i - \overline{x})}{S_{XX}}\sigma^2
  \end{split}
\end{equation*}
\EndKnitrBlock{proof}

\BeginKnitrBlock{proposition}[Properties of residuals]
\protect\hypertarget{prp:resprop}{}{\label{prp:resprop} \iffalse (Properties
of residuals) \fi{} }Mean and variance of the residual

\(\text{(a)}\: E(e_i) = 0\)

\(\text{(b)}\: Var(e_i) \neq \sigma^2\)

\(\text{(c)}\: \forall i \neq j : Cov(e_i, e_j) \neq 0\)
\EndKnitrBlock{proposition}

\BeginKnitrBlock{proof}
\iffalse{} {Proof. } \fi{}(a) Recall that this is the assumption of the
regression model.

\begin{enumerate}
\def\labelenumi{(\alph{enumi})}
\setcounter{enumi}{1}
\tightlist
\item
  Lemma \ref{lem:yandbet} implies that
\end{enumerate}

\begin{equation*}
  \begin{split}
    Cov(\overline{Y}, \hat\beta_1) & = Cov(\frac{1}{n}\sum Y_i, \hat\beta_1) \\
    & = \frac{1}{n} \sum_{i = 1}^n \frac{(x_i - \overline{x})}{S_{XX}}\sigma^2 \\
    & = 0 \qquad \because \sum (x_i - \overline{x}) = 0
  \end{split}
\end{equation*}

Then

\begin{equation}
  \begin{split}
    Var(\hat{Y_i}) & = Var(\hat\beta_0 + \hat\beta_1 x_i) \\
    & = Var \bigg[ \overline{Y} + (x_i - \overline{x}) \hat\beta_1 \bigg] \qquad \because \hat\beta_0 = \overline{Y} - \hat\beta_1 \overline{x} \\
    & = Var(\overline{Y}) + (x_i - \overline{x})^2 Var(\hat\beta_1) + 2(x_i - \overline{x}) Cov(\overline{Y}, \hat\beta_1) \\
    & = \frac{\sigma^2}{n} + (x_i - \overline{x})^2\frac{\sigma^2}{S_{XX}} + 0 \\
    & = \bigg( \frac{1}{n} + \frac{(x_i - \overline{x})^2}{S_{XX}} \bigg)\sigma^2
  \end{split}
  \label{eq:predvar}
\end{equation}

From the same lemma \ref{lem:yandbet},

\begin{equation}
  \begin{split}
    Cov(Y_i, \hat{Y_i}) & = Cov(Y_i, \overline{Y} + (x_i - \overline{x}) \hat\beta_1) \\
    & = Cov(Y_i, \overline{Y}) + (x_i - \overline{x}) Cov(Y_i, \hat\beta_1) \\
    & = \frac{\sigma^2}{n} + \frac{(x_i - \overline{x})^2}{S_{XX}}\sigma^2 \qquad \because Cov(Y_i, \hat\beta_1) = \frac{(x_i - \overline{x})}{S_{XX}}\sigma^2 \\
    & = \bigg( \frac{1}{n} + \frac{(x_i - \overline{x})^2}{S_{XX}} \bigg)\sigma^2
  \end{split}
  \label{eq:yyhat}
\end{equation}

These Equations \eqref{eq:predvar} and \eqref{eq:yyhat} give that

\begin{equation*}
  \begin{split}
    Var(e_i) & = Var(Y_i) + Var(\hat{Y_i}) -2Cov(Y_i, \hat{Y_i}) \\
    & = \sigma^2 + \bigg( \frac{1}{n} + \frac{(x_i - \overline{x})^2}{S_{XX}} \bigg)\sigma^2 - 2 \bigg( \frac{1}{n} + \frac{(x_i - \overline{x})^2}{S_{XX}} \bigg)\sigma^2 \\
    & = \bigg(1 - \frac{1}{n} - \frac{(x_i - \overline{x})^2}{S_{XX}} \bigg)\sigma^2 \\
    & \neq \sigma^2
  \end{split}
\end{equation*}

\begin{enumerate}
\def\labelenumi{(\alph{enumi})}
\setcounter{enumi}{2}
\tightlist
\item
  Let \(i \neq j\). Then
\end{enumerate}

\begin{equation*}
  \begin{split}
    Cov(e_i, e_j) & = Cov\Big( Y_i - (\hat\beta_0 + \hat\beta_1 x_i), Y_j - (\hat\beta_0 + \hat\beta_1 x_j) \Big) \\
    & = Cov(Y_i, Y_j) - Cov\Big(Y_i, (\hat\beta_0 + \hat\beta_1 x_j) \Big) - Cov((\hat\beta_0 + \hat\beta_1 x_i), Y_j) + Cov((\hat\beta_0 + \hat\beta_1 x_i), (\hat\beta_0 + \hat\beta_1 x_j)) \\
    & = 0 - \bigg( \frac{1}{n} - \frac{(x_i - \overline{x})\overline{x}}{S_{XX}} \bigg)\sigma^2 - \frac{(x_i - \overline{x})x_j}{S_{XX}}\sigma^2 \\
    & \qquad - \bigg( \frac{1}{n} - \frac{(x_j - \overline{x})\overline{x}}{S_{XX}} \bigg)\sigma^2 - \frac{(x_i - \overline{x})x_i}{S_{XX}}\sigma^2 \\
    & \qquad + \bigg( \frac{1}{n} + \frac{\overline{x}^2 + x_i x_j - \overline{x}(x_i + x_j)}{S_{XX}} \bigg)\sigma^2 \\
    & = - \bigg( \frac{1}{n} + \frac{\overline{x}^2 + x_i x_j - \overline{x}(x_i + x_j)}{S_{XX}} \bigg)\sigma^2 \\
    & \neq 0
  \end{split}
\end{equation*}
\EndKnitrBlock{proof}

\section{Decomposition of Total
Variability}\label{decomposition-of-total-variability}

\subsection{Total sum of squares}\label{total-sum-of-squares}

\BeginKnitrBlock{definition}[Uncorrected Total Sum of Squares]
\protect\hypertarget{def:unsst}{}{\label{def:unsst} \iffalse (Uncorrected
Total Sum of Squares) \fi{} }\[SST_{uncor} := \sum_{i = 1}^n Y_i^2\]
\EndKnitrBlock{definition}

\BeginKnitrBlock{definition}[Corrected Total Sum of Squares]
\protect\hypertarget{def:sst}{}{\label{def:sst} \iffalse (Corrected Total
Sum of Squares) \fi{} }\[SST := \sum_{i = 1}^n (Y_i - \overline{Y})^2\]
\EndKnitrBlock{definition}

What does this total sum of squares mean? To know this, we should know
\(\overline{Y}\) first.

\begin{Shaded}
\begin{Highlighting}[]
\NormalTok{delv }\OperatorTok\StringTok{ }
\StringTok{  }\KeywordTok{ggplot}\NormalTok{(}\KeywordTok{aes}\NormalTok{(}\DataTypeTok{x =}\NormalTok{ x, }\DataTypeTok{y =}\NormalTok{ y)) }\OperatorTok{+}
\StringTok{  }\KeywordTok{geom_smooth}\NormalTok{(}\DataTypeTok{method =} \StringTok{"lm"}\NormalTok{, }\DataTypeTok{formula =}\NormalTok{ y }\OperatorTok{~}\StringTok{ }\DecValTok{1}\NormalTok{, }\DataTypeTok{se =} \OtherTok{FALSE}\NormalTok{) }\OperatorTok{+}
\StringTok{  }\KeywordTok{geom_point}\NormalTok{() }\OperatorTok{+}
\StringTok{  }\KeywordTok{labs}\NormalTok{(}
    \DataTypeTok{x =} \StringTok{"Number of Cases"}\NormalTok{,}
    \DataTypeTok{y =} \StringTok{"Delivery Time"}
\NormalTok{  )}
\end{Highlighting}
\end{Shaded}

\begin{figure}[H]

{\centering \includegraphics[width=0.7\linewidth]{regression-analysis_files/figure-latex/ybarpred-1} 

}

\caption{Regression without predictor}\label{fig:ybarpred}
\end{figure}

See Figure \ref{fig:ybarpred}. The line represents the closest line when
we use only intercept term for the regression model. In other words,
\emph{if we use no information for the response}, i.e.~no predictor
variables, we will get just average of the response variable. Consider

\[Y_i = \beta_0 + \epsilon_i\]

Then we can get only one normal equation

\[\sum (Y_i - \hat\beta_0) = 0\]

Hence,

\[\hat\beta_0 = \frac{1}{n} \sum_{i = 1}^n Y_i \equiv \overline{Y}\]

From this fact, \(SST\) implies \textbf{total variance}.

\subsection{Regression sum of squares}\label{regression-sum-of-squares}

\BeginKnitrBlock{definition}[Regression Sum of Squares]
\protect\hypertarget{def:ssr}{}{\label{def:ssr} \iffalse (Regression Sum of
Squares) \fi{} }\[SSR := \sum_{i = 1}^n (hat{Y_i} - \overline{Y})^2\]
\EndKnitrBlock{definition}

This \(SSR\) compares \(\hat{Y_i}\) versus \(\overline{Y}\), computing
the sum of squares for difference between predicted values from
\emph{regression model} and \emph{model not using predictors}.

\subsection{Residual sum of squares}\label{residual-sum-of-squares}

Now consider the \emph{residual sum of squares} \(SSE\) in the
definition \ref{def:sse}. As mentioned, this is related to the
\emph{prediction errors}, which the regression model could not explain
the data.

\subsection{Decomposition of total sum of squares}\label{decompsst}

\(SST\) can be decomposed by construction of sum of squares.

\BeginKnitrBlock{proposition}[Decomposition of SST]
\protect\hypertarget{prp:decom}{}{\label{prp:decom} \iffalse (Decomposition
of SST) \fi{} }\[SST = SSR + SSE\]

where \(SST = \sum (Y_i - \overline{Y})^2\),
\(SSR = \sum (\hat{Y_i} - \overline{Y})^2\), and
\(SSE = \sum (Y_i - \hat{Y_i})\)
\EndKnitrBlock{proposition}

\BeginKnitrBlock{proof}
\iffalse{} {Proof. } \fi{}From the Example \ref{exm:usingnormal},

\begin{equation*}
  \begin{split}
    \sum_{i = 1}^n (Y_i - \overline{Y})^2 & = \sum_{i = 1}^n (Y_i - \hat{Y_i} + \hat{Y_i} - \overline{Y})^2 \\
    & = \sum_{i = 1}^n (Y_i - \hat{Y_i})^2 + 2 \sum_{i = 1}^n (Y_i - \hat{Y_i})(\hat{Y_i} - \overline{Y}) + \sum_{i = 1}^n (\hat{Y_i} - \overline{Y})^2 \\
    & = \sum_{i = 1}^n (Y_i - \hat{Y_i})^2 + \sum_{i = 1}^n (\hat{Y_i} - \overline{Y})^2 \qquad \because \sum (Y_i - \hat{Y_i}) = 0 \: \text{and} \: \sum (Y_i - \hat{Y_i})\hat{Y_i} = 0
  \end{split}
\end{equation*}
\EndKnitrBlock{proof}

This represents each \(SSR\) and \(SSE\) divides total variability as
following.

\[\overset{SST}{\text{total variability}} = \overset{SSR}{\text{left unexplained by regression}} + \overset{SSE}{\text{explained by regression}}\]

Denote that the total variability \(SST\) is \emph{constant given data
set}. If our model is good, \(SSR\) grows and \(SSE\) flattens. Thus the
larger \(SSR\) is, the better. The lower \(SSE\) is, the better.

\subsection{Coefficient of
determination}\label{coefficient-of-determination}

We have discussed in the previous section \ref{decompsst} that \(SSR\)
and \(SSE\) splits the total variability into \emph{explained part and
not-explained part by our regression model}. Our first interest is
whether the model works well for the data well, so we can think about
the \emph{proportion of explained part to the total variance}. The
following measure \(R^2\) computes this kind of value.

\BeginKnitrBlock{definition}[Coefficient of Determination]
\protect\hypertarget{def:rsq}{}{\label{def:rsq} \iffalse (Coefficient of
Determination) \fi{}
}\[R^2 := \frac{SSR}{SST} = 1 - \frac{1 - SSE}{SST}\]
\EndKnitrBlock{definition}

By construction,

\[0 \le R^2 \le 1\]

As \(R^2\) goes to \(0\), the model goes wrong. As \(R^2\) is close to
\(1\), large proportion of variability has been explained. So we prefer
large values rather than small.

\BeginKnitrBlock{proposition}
\protect\hypertarget{prp:rsqlin}{}{\label{prp:rsqlin} }\(R^2\) shows the
strength of linear relation between two variables \(x\) and \(Y\) in the
simple linear regression.

\[R^2 = \hat\rho_{XY}\]

where
\(\hat\rho_{XY} := \frac{\sum (X_i - \overline{X})(Y_i - \overline{Y})}{\sqrt{\sum (X_i - \overline{X})^2} \sqrt{\sum (Y_i - \overline{Y})^2}}\)
is the sample correlation coefficients
\EndKnitrBlock{proposition}

\BeginKnitrBlock{proof}
\iffalse{} {Proof. } \fi{}Note that
\(\hat{Y_i} - \overline{Y} = \hat\beta_1 (x_i - \overline{x}) = \frac{S_{XY}}{S_{XX}} (x_i - \overline{x})\).
Then

\begin{equation*}
  \begin{split}
    \sum (\hat{Y_i} - \overline{Y})^2 & = \frac{S_{XY}^2}{S_{XX}^2} \sum (x_i - \overline{x})^2 \\
    & = \frac{S_{XY}^2}{S_{XX}}
  \end{split}
\end{equation*}

It follows that

\begin{equation*}
  \begin{split}
    R^2 & = \frac{\sum (\hat{Y_i} - \overline{Y})^2}{\sum (Y_i - \overline{Y})^2} \\
    & = \frac{S_{XY}^2}{S_{XX}S_{YY}} \\
    & =: \hat\rho_{XY}^2
  \end{split}
\end{equation*}
\EndKnitrBlock{proof}

In this relation, we can know that \(R^2\) statistic performs as a
measure of the linear relationship in the simple linear regression
setting.

\section{Geometric Interpretations}\label{geometric-interpretations}

\subsection{Fundamental subspaces}\label{fundamental-subspaces}

These linear algebra concepts might be more useful for \emph{multiple
linear regression}, but let's briefly recap \citep{Leon:2014aa}.

\BeginKnitrBlock{definition}[Fundamental Subspaces]
\protect\hypertarget{def:subspace}{}{\label{def:subspace}
\iffalse (Fundamental Subspaces) \fi{} }Let
\(X \in \mathbb{R}^{n \times (p + 1)}\).

Then the Null space is defined by

\[N(X) := \{ \mathbf{b} \in \mathbb{R}^n \mid X\mathbf{b} = \mathbf{0} \}\]

The Row space is defined by

\[Row(X) := sp(\{\mathbf{r}_1, \ldots, \mathbf{r}_{p + 1} \}) \quad \text{where} \: X^T = [\mathbf{r}_1^T, \ldots, \mathbf{r}_{n}^T]\]

The Column space is defined by

\[Col(X) := sp(\{\mathbf{c}_1, \ldots, \mathbf{c}_{n} \}) \quad \text{where} \: X = [\mathbf{c}_1, \ldots, \mathbf{c}_{p + 1}]\]

The Range of \(X\) is defined by

\[R(X) := \{ \mathbf{y} \in \mathbb{R}^n \mid \mathbf{y} = X\mathbf{b} \quad \text{for some} \: \mathbf{b} \in \mathbb{R}^{p + 1} \}\]
\EndKnitrBlock{definition}

These spaces have some constructional relationship.

\BeginKnitrBlock{theorem}[Fundamental Subspaces Theorem]
\protect\hypertarget{thm:fundsub}{}{\label{thm:fundsub}
\iffalse (Fundamental Subspaces Theorem) \fi{} }Let
\(X \in \mathbb{R}^{n \times (p + 1)}\). Then

\[N(X) = R(X^T)^{\perp} = Col(X^T)^{\perp} = Row(X)^{\perp}\]

Transposed matrix also satisfy this.

\[N(X^T) = R(X)^{\perp} = Col(X)^{\perp}\]
\EndKnitrBlock{theorem}

\BeginKnitrBlock{proof}
\iffalse{} {Proof. } \fi{}Let \(\mathbf{a} \in N(X)\). Then
\(X\mathbf{a} = \mathbf{0}\).

Let \(\mathbf{y} \in R(X^T)\). Then \(X^T \mathbf{b} = \mathbf{y}\) for
some \(\mathbf{b} \in \mathbb{R}^{p + 1}\).

Choose \(\mathbf{b} \in \mathbb{R}^{p + 1}\) such that
\(X^T \mathbf{b} = \mathbf{y}\). Then

\begin{equation*}
  \begin{split}
    \mathbf{0} & = X\mathbf{a} \\
    & = \mathbf{b}^T X\mathbf{a} \\
    & = \mathbf{y}^T \mathbf{a}
  \end{split}
\end{equation*}

Hence,

\[N(X) \perp R(X^T)\]

Since

\[X^T \mathbf{b} = \mathbf{c}_1 \mathbf{b} + \cdots + \mathbf{c}_{p + 1} \mathbf{b}\]

it is trivial that \(R(X) = Col(X)\) and \(R(X^T) = Col(X^T)\).

If \(\mathbf{a} \in N(X)\), then

\[
X\mathbf{a} = \begin{bmatrix}
  \mathbf{r}_1 \\
  \mathbf{r}_2 \\
  \cdots \\
  \mathbf{r}_n
\end{bmatrix} \begin{bmatrix}
  a_1 \\
  \cdots \\
  a_{p + 1}
\end{bmatrix} = \begin{bmatrix}
  0 \\
  0 \\
  \cdots \\
  0
\end{bmatrix}
\]

Thus,

\[\forall i :  \mathbf{a}^T \mathbf{r}_i = 0\]

and so

\[N(X) \subseteq Row(X)^{\perp}\]

Conversely, if \(\mathbf{a} \in Row(X)^{\perp}\), then
\(\forall i : \mathbf{a}^T \mathbf{r}_i = 0\). This implies that
\(X\mathbf{a} = \mathbf{0}\). Thus,

\[Row(X)^{\perp} \subseteq N(X)\]

and so

\[N(X) = Row(X)^{\perp}\]
\EndKnitrBlock{proof}

\(N(X^T) = R(X)^{\perp}\) part in Theorem \ref{thm:fundsub} will give
the geometric insight to \emph{least squares solution}.

\BeginKnitrBlock{theorem}
\protect\hypertarget{thm:perpbasis}{}{\label{thm:perpbasis} }Let \(S\) be a
subspace of \(\mathbb{R}^n\). Then

\[dim S + dim S^{\perp} = n\]

If \(\{ \mathbf{x}_1, \ldots, \mathbf{r} \}\) is a basis for \(S\) and
\(\{ \mathbf{x}_{r + 1}, \ldots, \mathbf{n} \}\) is a basis for
\(S^{\perp}\), then
\(\{ \mathbf{x}_1, \ldots, \mathbf{x}_r, \mathbf{x}_{r + 1}, \ldots, \mathbf{n} \}\)
is a basis for \(\mathbb{R}^n\).
\EndKnitrBlock{theorem}

\BeginKnitrBlock{theorem}
\protect\hypertarget{thm:dsum}{}{\label{thm:dsum} }Let \(S\) be a subspace
of \(\mathbb{R}^n\). Then

\[\mathbb{R}^n = S \oplus S^{\perp}\]
\EndKnitrBlock{theorem}

\subsection{Simple linear regression}\label{simple-linear-regression}

\BeginKnitrBlock{theorem}
\protect\hypertarget{thm:projection}{}{\label{thm:projection} }Let \(S\) be
a subspace of \(\mathbb{R}^n\). For each
\(\mathbf{y} \in \mathbf{R}^n\), there exists a unique
\(\mathbf{p} \in S\) that is closest to \(\mathbf{y}\), i.e.

\[\Vert \mathbf{y} - \mathbf{p}  \Vert > \Vert \mathbf{y} - \mathbf{\hat{y}} \Vert\]

for any \(\mathbf{p} \neq \mathbf{\hat{y}}\). Furthermore, a given
vector \(\mathbf{p} \in S\) will be the closest to a given vector
\(\mathbf{y} \in \mathbb{R}^n\) if and only if

\[\mathbf{y} - \mathbf{\hat{y}} \in S^{\perp}\]
\EndKnitrBlock{theorem}

Least square estimator \((\hat\beta_0, \hat\beta_1)^T\) minimizes

\[\sum_{i = 1}^n (Y_i - \beta_0 - \beta_1 x_i)^2 = \Vert \mathbf{Y} - (\beta_0 \mathbf{1} + \beta_1 \mathbf{x}) \Vert^2\]

with respect to \((\hat\beta_0, \hat\beta_1)^T \in \mathbb{R}^2\) (where
\(\mathbf{1} := (1, 1)^T\)). Recall that the normal equation gives

\[\sum_{i = 1}^n(Y_i - \hat\beta_0 - \hat\beta_1 x_i) = \Big( \mathbf{Y} - (\hat\beta_0 \mathbf{1} + \hat\beta_1 \mathbf{x}) \Big)^T \mathbf{1} = 0\]

and

\[\sum_{i = 1}^n (Y_i - \hat\beta_0 - \hat\beta_1 x_i)x_i = \Big( \mathbf{Y} - (\hat\beta_0 \mathbf{1} + \hat\beta_1 \mathbf{x}) \Big)^T \mathbf{x} = 0\]

These two relation give

\[\mathbf{Y} - (\hat\beta_0 \mathbf{1} + \hat\beta_1 \mathbf{x}) \perp sp(\{ \mathbf{1}, \mathbf{x} \})^{\perp}\]

i.e.
\(\mathbf{\hat{Y}} = \hat\beta_0 \mathbf{1} + \hat\beta_1 \mathbf{x}\)
is the projection of \(\mathbf{Y}\).

Theorem \ref{thm:projection} can give the same result.

\[\hat\beta_0 \mathbf{1} + \hat\beta_1 \mathbf{x} \in R([\mathbf{1}, \mathbf{x}])^{\perp} = sp(\{ \mathbf{1}, \mathbf{x} \})^{\perp}\]

\begin{figure}[H]

{\centering \includegraphics[width=0.7\linewidth]{regression-analysis_files/figure-latex/simpledraw-1} 

}

\caption{Geometric Illustration of Simple Linear Regression}\label{fig:simpledraw}
\end{figure}

We can see the details from Figure \ref{fig:simpledraw}. In fact,
decomposition of \(SST\) and \(R^2\) are also in here.

\begin{figure}[H]

{\centering \includegraphics[width=0.7\linewidth]{regression-analysis_files/figure-latex/simpledraw2-1} 

}

\caption{Geometric Illustration of Decomposing SST}\label{fig:simpledraw2}
\end{figure}

See Figure \ref{fig:simpledraw2}.

\[
\begin{cases}
  SST = \lVert \mathbf{Y} - \overline{Y} \mathbf{1} \rVert^2 \\
  SSR = \lVert \mathbf{\hat{Y}} - \overline{Y} \mathbf{1} \rVert^2 \\
  SSE = \lVert \mathbf{Y} - \mathbf{\hat{Y}} \rVert^2
\end{cases}
\]

Pythagorean law implies that

\[SST = SSR + SSE\]

Also,

\[R^2 = \frac{SSR}{SST} = cos^2\theta = \hat\rho_{XY}^2\]

\section{Distributions}\label{distributions}

\subsection{Mean response and
response}\label{mean-response-and-response}

We have already look at predicting each mean response and response from
equation \eqref{eq:meanres} and \eqref{eq:indpred}.

\BeginKnitrBlock{theorem}[Estimation of the mean response]
\protect\hypertarget{thm:mux}{}{\label{thm:mux} \iffalse (Estimation of the
mean response) \fi{}
}\[\hat\mu_x \equiv \widehat{E(Y \mid x)} = \hat\beta_0 + \hat\beta_1 x\]
\EndKnitrBlock{theorem}

\BeginKnitrBlock{theorem}[(out of sample) Prediction of a response]
\protect\hypertarget{thm:yhatx}{}{\label{thm:yhatx} \iffalse ((out of
sample) Prediction of a response) \fi{}
}\[\hat{Y_x}  = \hat\beta_0 + \hat\beta_1 x\]
\EndKnitrBlock{theorem}

Recall that predicting \ref{thm:mux} targets at

\[\mu_x \equiv E(Y \mid x) = \beta_0 + \beta_1 x\]

which have been assumed to be true model. On the other hand, predicting
\ref{thm:yhatx} targets at

\[Y = \beta_0 + \beta_1 + \epsilon_x\]

The linearity is not true in reality. So the errors caused by modeling
linear model are included in \(\epsilon_x\). This error term makes
difference between properties of \ref{thm:mux} and \ref{thm:yhatx}.

To derive their distribution and see the difference, we additionaly
assume Normality, i.e.

\[\epsilon_i \stackrel{iid}{\sim} N(0, \sigma^2)\]

\subsection{Regression coefficients}\label{regression-coefficients}

Under Normality, we have

\[Y_i \stackrel{indep}{\sim} N(\beta_0 + \beta_1 x_i, \sigma^2)\]

Then

\[
\mathbf{Y} = \begin{bmatrix}
  Y_1 \\
  Y_2 \\
  \vdots \\
  Y_n
\end{bmatrix} \sim MVN_n\Bigg( \boldsymbol\mu \equiv \begin{bmatrix}
  \beta_0 + \beta_1 x_1 \\
  \beta_0 + \beta_1 x_2 \\
  \vdots \\
  \beta_0 + \beta_1 x_n
\end{bmatrix}, \Sigma \equiv \sigma^2 I = \begin{bmatrix}
  \sigma^2 & 0 & \cdots & 0 \\
  0 & \sigma^2 & \cdots & 0 \\
  \vdots & \vdots & \vdots & \vdots \\
  0 & 0 & 0 & \sigma^2
\end{bmatrix} \Bigg)
\]

Write \(\boldsymbol{\hat\beta} = (\hat\beta_0, \hat\beta_1)^T\). From
Lemma \ref{lem:linbet},

\[\hat\beta_0 = \mathbf{a}^T\mathbf{Y}\]

where \(\mathbf{a} = (a_1, \ldots, a_n)^T \in \mathbb{R}^n\) with
\(a_i = \bigg( \frac{1}{n} - \frac{(x_i - \overline{x})\overline{x}}{S_{XX}} \bigg)\)

and

\[\hat\beta_1 = \mathbf{b}^T\mathbf{Y}\]

where \(\mathbf{b} = (b_1, \ldots, b_n)^T \in \mathbb{R}^n\) with
\(b_i = \frac{(x_i - \overline{x})}{S_{XX}}\).

Let

\[A^T = [ \mathbf{a}^T, \mathbf{b}^T ]\]

Then

\[
\boldsymbol{\hat\beta} = A\mathbf{Y}
\]

Linearity of the multivariate normal distribution, Proposition
\ref{prp:ue} and \ref{prp:vb} imply that

\begin{equation}
  \boldsymbol{\hat\beta} = \begin{bmatrix}
    \hat\beta_0 \\ \hline
    \hat\beta_1
  \end{bmatrix} \sim MVN \bigg( A\boldsymbol\mu = \begin{bmatrix}
    \beta_0 \\ \hline
    \beta_1
  \end{bmatrix},
  A\Sigma A^T = \sigma^2 AA^T = \begin{bmatrix}
    \bigg( \frac{1}{n} + \frac{\overline{x}^2}{S_{XX}} \bigg)\sigma^2 & - \frac{\overline{x}}{S_{XX}}\sigma^2 \\ \hline
    - \frac{\overline{x}}{S_{XX}}\sigma^2 & \frac{\sigma^2}{S_{XX}}
  \end{bmatrix} \bigg)
  \label{eq:b01mvn}
\end{equation}

Hence,

\begin{equation}
  \hat\beta_0 \sim N \bigg( \beta_0, \bigg( \frac{1}{n} + \frac{\overline{x}^2}{S_{XX}} \bigg)\sigma^2 \bigg)
  \label{eq:b0dist}
\end{equation}

\begin{equation}
  \hat\beta_1 \sim N \bigg( \beta_1, \frac{\sigma^2}{S_{XX}} \bigg)
  \label{eq:b1dist}
\end{equation}

\subsection{Mean response}\label{mean-response}

\bibliography{book.bib,packages.bib}


\end{document}
